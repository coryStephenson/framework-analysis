


\begin{table}[h!]
    \renewcommand*{\arraystretch}{2}
    \centering
    \scriptsize
    \noindent\makebox[\textwidth]{
    \begin{tabularx}{1.3\textwidth}{>{\columncolor{blue}\color{white}\bfseries}P{65mm}P{65mm}P{65mm}}
        \rowcolor[gray]{0.8}
        \color{black} Distinguishing Features & \bfseries CIS CSC & \bfseries NIST CSF \\[2pt]
        Comprehensiveness vs.\ Prescriptiveness & CIS Controls are renowned for their prescriptive nature, offering meticulous, actionable guidance.
& NIST Framework casts a broader net, enabling organizations to tailor cybersecurity measures to their specific requirements.
 \\
        Implementation Groups vs.\ Implementation Tiers & CIS deploys Implementation Groups to categorize organizations based on size and cybersecurity maturity.
& NIST relies on Implementation Tiers to gauge organizations' cybersecurity readiness.
 \\
        Specificity vs.\ Flexibility & CIS Controls are specific, leaving minimal room for interpretation, providing organizations with clear directives.
& NIST Framework is flexible, allowing organizations to adapt guidelines to their unique needs.
 \\
        When to Choose NIST Over CIS & While CIS is a strong contender, there are scenarios where NIST is the preferred choice.
& \begin{enumerate}[leftmargin=10pt, labelindent=0pt, itemindent=0pt] \item \textbf{Government Contracts:} Organizations involved in government contracts or the federal supply chain must comply with NIST standards, such as NIST Special Publications 800-171 and 800-53.
 \item \textbf{Mature Security Postures:} NIST frameworks are well-suited for organizations with mature security policies and a clear understanding of their cybersecurity needs.
 \item \textbf{Customization:} NIST's flexibility allows for tailoring cybersecurity measures to fit an organization's unique resources and objectives.
 \end{enumerate} \\
    When to Choose CIS Over NIST & \begin{enumerate}[leftmargin=10pt, labelindent=0pt, itemindent=0pt] \item \textbf{Practical Implementation:} For organizations seeking actionable, step-by-step guidance for implementing cybersecurity controls, CIS offers a clear advantage.
\item \textbf{Cross-Functional Teams:} The common language used in CIS documentation facilitates communication between technical and non-technical teams working on security initiatives.
\item \textbf{Framework of Frameworks:} CIS incorporates elements from various frameworks, including NIST, providing a consolidated approach for organizations lacking a comprehensive security policy.
\end{enumerate} & While NIST is a strong contender, there are scenarios where NIST is the preferred choice. \\
Use Cases of CIS vs. NIST & CIS controls are well-suited for organizations seeking to implement security controls effectively. They are particularly beneficial for those without a comprehensive security policy and unsure about implementation prioritization.
& NIST frameworks shine in diagnostics, organization, and planning. They are ideal for mature organizations looking to enhance their existing security policies and cater to specific regulatory requirements. \\
Cross-Compatibility with Compliance Frameworks & Both CIS and NIST frameworks align with various cybersecurity standards and compliance frameworks, but CIS does a better job of mapping into different standards. This makes CIS a valuable starting point for complying with regulations like PCI DSS, HIPAA, GDPR, and ISO 27001. & \\
Coverage Comparison & While CIS offers practical recommendations, NIST documentation is more comprehensive. It covers a wide range of cybersecurity standards, including NIST SP 800-53, 800-171, 800-37, and others, making it suitable for organizations aiming for high-level contracts or working with sensitive government data. & \\
\end{tabularx}
}

\caption{CIS CSC vs.\ NIST CSF Comparison Table}
\end{table}

